%%%%%%%%%%%%%%%%%%%%%%%%%%%%%%%%%%%%%%%%%%%%%%%%%%%%%%%%%%%%%%%%%%%%%%%%%%%%%%%%
%2345678901234567890123456789012345678901234567890123456789012345678901234567890
%        1         2         3         4         5         6         7         8

\documentclass[letterpaper, 10 pt, conference]{ieeeconf}  % Comment this line out
                                                          % if you need a4paper
%\documentclass[a4paper, 10pt, conference]{ieeeconf}      % Use this line for a4
                                                          % paper

\IEEEoverridecommandlockouts                              % This command is only
                                                          % needed if you want to
                                                          % use the \thanks command
\overrideIEEEmargins
% See the \addtolength command later in the file to balance the column lengths
% on the last page of the document

\usepackage[utf8]{inputenc}
\usepackage[T1]{fontenc}

% The following packages can be found on http:\\www.ctan.org
%\usepackage{graphics} % for pdf, bitmapped graphics files
%\usepackage{epsfig} % for postscript graphics files
%\usepackage{mathptmx} % assumes new font selection scheme installed
%\usepackage{mathptmx} % assumes new font selection scheme installed
%\usepackage{amsmath} % assumes amsmath package installed
%\usepackage{amssymb}  % assumes amsmath package installed

\title{\LARGE \bf
    Historia de las Redes de Transmisión de Datos e Internet
}

%\author{ \parbox{3 in}{\centering Huibert Kwakernaak*
%         \thanks{*Use the $\backslash$thanks command to put information here}\\
%         Faculty of Electrical Engineering, Mathematics and Computer Science\\
%         University of Twente\\
%         7500 AE Enschede, The Netherlands\\
%         {\tt\small h.kwakernaak@autsubmit.com}}
%         \hspace*{ 0.5 in}
%         \parbox{3 in}{ \centering Pradeep Misra**
%         \thanks{**The footnote marks may be inserted manually}\\
%        Department of Electrical Engineering \\
%         Wright State University\\
%         Dayton, OH 45435, USA\\
%         {\tt\small pmisra@cs.wright.edu}}
%}

\author{ Cristian Camilo Serna Betancur, Elizabeth Pérez Alfonso y Brian Vanegas Parra}


\begin{document}



\maketitle
\thispagestyle{empty}
\pagestyle{empty}


%%%%%%%%%%%%%%%%%%%%%%%%%%%%%%%%%%%%%%%%%%%%%%%%%%%%%%%%%%%%%%%%%%%%%%%%%%%%%%%%
%\begin{abstract}

%This electronic document is a ``live'' template. The various components of your paper [title, text, heads, etc.] are already defined on the style sheet, as illustrated by the portions given in this document.

%\end{abstract}


%%%%%%%%%%%%%%%%%%%%%%%%%%%%%%%%%%%%%%%%%%%%%%%%%%%%%%%%%%%%%%%%%%%%%%%%%%%%%%%%
\section{Origen de Internet}

En 1969, el Ministerio de Defensa de los EE.UU. creó una red llamada ARPANET con el objetivo de conectar a sus investigadores con centros de cálculo lejanos, permitiéndoles compartir recursos que no tenían en sus propios ordenadores. El acceso a esta red se limitó durante los primeros años a los militares, a las empresas de armamento y a las universidades que realizan investigación sobre defensa. También se conectaron a esta red otras redes experimentales. Este conjunto de redes interconectadas, dedicadas a la investigación militar, fue lo que se llamó por primera vez Internet.

\section{Servicios sobre la Internet}

Aunque usamos Web e Internet como sinónimos, no hay que confundirlos, son cosas diferentes. La www es solamente uno de los servicios más conocidos que presta Internet, entre muchos otros, que conoceremos a continuación: 

World Wide Web
Correo Electrónico
Internet Relay Chat
Streaming
Telefonía VoIP
Transferencia de Archivos


\section{ARPANET}

La Advanced Research Projects Agency (ARPA), o en español, Agencia de Proyectos de Investigación Avanzada, fue una iniciativa del Departamento de Defensa de USA creada en 1958; la cual tenía dentro de sus objetivos, tener un medio de comunicación seguro entre los diferentes organismos del Estado. 

En 1962 ARPA creó un programa de investigación computacional dirigido por Joseph C. R. Licklider, un investigador del MIT. Él tuvo la idea de crear una red de computadores que fuera capaz de comunicar usuarios en distintos computadores. Aunque Licklider abandonó el proyecto junto con ARPA en 1964, Robert Taylor tuvo brillantes ideas basadas en la propuesta de Licklider y el proyecto continúo. Así fue que en 1967 el ARPA publicó un plan para crear una red de ordenadores denominada ARPANET, acrónimo de Advanced Research Projects Agency Network.

En 1969 fue enviado el primer mensaje entre dos computadores, ubicados en la Universidad de California y en el Stanford Research Institute de California, a través de la red ARPANET, sin embargo ésta falló, logrando comunicar solo los dos primeros caracteres de la palabra “login”. Una hora después logró hacer el envío completo, siendo esa la primera vez que un ordenador se conecta con otro a cientos de kilómetros de distancia. Por lo tanto el origen de Internet proviene de las investigaciones universitarias y del Ministerio de Defensa de Estados Unidos.

\section{TCP/IP}

El modelo TCP/IP es una descripción de protocolos de red, en ocasiones se le denomina conjunto de protocolos TCP/IP en referencia a los dos protocolos más importantes que la componen. Donde TCP o protocolo de control de transmisión, puede ser usado por los programas dentro de una red de datos compuesta por una red de ordenadores para crear conexiones entre sí a través de las cuales puede enviarse un flujo de datos e IP o protocolo de internet, identifica de manera exclusiva un nodo de internet (en una red de computadoras, cada una de las máquinas es un nodo). 

\section{Usenet}

Usenet es un término derivado de Users Network (red de usuarios), fue una red creada en 1979 y comenzó a funcionar en 1980. Es uno de los sistemas más antiguos de comunicaciones entre redes de computadoras, aún en uso. Fue originalmente creada para compartir noticias entre los usuarios de Unix, pero actualmente se utiliza principalmente para compartir o descargar archivos. 

\section{ICANN (Internet Corporation for Assigned Names and Numbers)}

Es la corporación encargada de administrar el sistema de nombres de dominio para garantizar la estabilidad operativa de Internet. Es una compañía sin ánimo de lucro creada en California (EEUU)  y su función principal es mejorar la gestión técnica de nombres y direcciones de Internet (Libro verde) y la Gestión de nombres y direcciones de Internet de la NTIA ( Papel blanco).

\section{IANA (Internet Assigned Numbers Authority)}

Es una división de ICANN, es la responsable de asignar / administrar direcciones IP, algunas cosas del dominio, números de puerto y la base de datos de la zona horaria, se encargan de coordinar la asignación de parámetros de protocolo previstos en los estándares técnicos de Internet y administrar el DNS.

\section{NIC (Network Information Center)}

Es un operador de registro que mantiene los datos administrativos de un dominio y genera un archivo de zona que contiene las direcciones de los servidores de nombres para cada dominio.

\section{InterNIC (Internet Network Information Center)}

Fue responsable de registrar y mantener los dominios de nivel superior .com, .net, .org... InterNIC actúa como un registro que mantiene una base de datos de información de dominio de nivel superior y un registrador que suministra servicios de registro de nombres.

\section{Gopher}

Gopher en esencia es un sistema de indexación y recuperación de archivos que utiliza el protocolo de Internet en los sistemas conectados. Solo puede acceder a información dentro de un sistema. Fue lanzado en 1991, Permitió a las personas buscar rápidamente recursos en Internet, cuando utilizaba un cliente Gopher, veía un menú de enlaces que conducían a documentos, aplicaciones, sitios FTP, etc.

\end{document}
